\documentclass[a4paper,10pt]{article}
\usepackage[]{graphicx}
\usepackage[]{times}
\usepackage{geometry}

\geometry{verbose,a4paper,tmargin=1.5cm,bmargin=2cm,lmargin=2cm,rmargin=2cm}
\renewcommand{\baselinestretch}{1.2}

\title{CADI installation manual \\ \small (version beta)}

\author{
GICI group \vspace{0.1cm} \\
\small Department of Information and Communications Engineering \\
\small Universitat Aut{\`o}noma Barcelona \\
\small http://www.gici.uab.es  -  http://www.gici.uab.es/CADI \\
}

\date{September 2007}

\hyphenation{CADIServer}
\hyphenation{CADIClient}
\hyphenation{CADIViewer}

\begin{document}
\maketitle

	CADI is programmed using the JAVA language. The used compiler is the
	SUN JAVA 1.5. Hence, to run the application the Runtime Environment
	is necessary. CADIViewer can save the recovered image in different
   image types CADI, so it uses 
	the JAI (Java Advanced Imaging) library. Thus, if you need to
	manage different image formats apart of raw images you should
	install it. This software can be freely downloaded from
	\emph{http://java.sun.com}. 

	CADI is provided with a single jar file (\emph{dist/CADI.jar}), that
	contains the following two applications: CADIServer, CADIViwer. To run one
	of these applications you can use the following command: 
	\emph{java-classpath dist/CADIServer.java} and 
	\emph{java-classpath dist/CADIViewer.java}
	applications. In a GNU/Linux environment you can also use the shell
	scripts \emph{CADIServer}, \emph{CADIViewer}, \ldots situated at the
	root of the CADI directory. 

	To install CADI as a system program you can follow the following
	steps: 

	\begin{enumerate}
		\item Decompress the CADI distribution and locate the
		\emph{CADIServer.jar} and	\emph{CADIViewer.jar} files in some directory,
		for example in \emph{/usr/local/CADI/CADIServer.jar} and
		\emph{/usr/local/CADI/CADIViewer.jar}.  
		\item Create the following shell script (for windows
		environments you can create a bat file): 
			\begin{verbatim}
			#!/bin/sh
			java -classpath /usr/local/CADI/CADIServer.jar CADI.CADIServer $@ 
			\end{verbatim}
			Rename this file and call it CADIServer.
		\item Create another shell script (or bat file) for the client
		called \emph{CADIViewer}. 
		\item Put both shell scripts in some directory included in your
		execution PATH variable. 
		\item To execute the server or the client run \emph{CADIServer} or
		\emph{CADIViewer} from your shell. 
	\end{enumerate}

\end{document}
